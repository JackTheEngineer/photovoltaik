\section{Berechnung der Messunsicherheiten}
Unter Verwendung der
gaußschen Fehlerfortpflanzung \ref{equ:staterr} wurden die
Fehler der Fits mit
\begin{equation}
	\sigma_{f(x_1,x_2,...)} = \sum\limits_{i=1}^n \abs{\frac{\partial f}{\partial x_i}\cdot \sigma_{x_i}}
	\label{equ:syserr}
\end{equation}
\begin{equation}
	\sigma_{f(x_1,x_2,...)} = \sqrt{\sum\limits_{i=1}^n (\frac{\partial f}{\partial x_i} \cdot \sigma_{x_i})^2}
	\label{equ:staterr}
\end{equation}
errechnet.
Ensprechend wurden die Längen der Fehlerbalken ermittelt.
Für die Berechnungen und das Plotten der Graphen wurde ein Pythonskript geschrieben.
Dieses kann auf der Webseite \url{https://github.com/JackTheEngineer/photovoltaik}
im Ordner \texttt{calc/} eingesehen werden.

