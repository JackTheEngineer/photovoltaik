\section{Questions}
The question why the quantum efficiency goes down for wavelengths of around can be answered in many ways. \newline
For the a-silicon for example one reason might be that the glass covering the silicon isn't as transparent for ultraviolet light as it is for light with higher wavelengths so there simply isn't as much light that can produce photocurrent inside of the silicon as for higher wavelengths. \newline
On the other hand for the c-silicon one finds after looking at \ref{fig:csiflat} again, that it's reflectivity for light in this range of wavelengths is particularly high, so the major part of the light hitting the silicon is reflected right back and therefore there isn't that much light to produce photocurrent left as for higher wavelengths. 
Another reason, that applies for both cells, is that the cells simply aren't thick enough to use these rather high energy photons. This means that the energy of the photons hitting the cells is just so high that they pass right through the cell without or just barely interacting with the atoms in the solar cell hence no photocurrent is produced and the quantum efficiency falls for these wavelengths. 