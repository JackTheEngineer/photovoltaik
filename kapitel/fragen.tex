\section{Questions}
The question why the quantum efficiency goes down for wavelengths of around can be answered in many ways. \newline
For the a-silicon for example one reason might be that the glass covering the silicon isn't as transparent for ultraviolet light as it is for light with higher wavelengths so there simply isn't as much light that can produce photocurrent inside of the silicon as for higher wavelengths. \newline
On the other hand for the c-silicon one finds after looking at \ref{fig:csiflat} again, that it's reflectivity for light in this range of wavelengths is particularly high, so the major part of the light hitting the silicon is reflected right back and therefore there isn't that much light to produce photocurrent left as for higher wavelengths. 
Another reason, that applies for both cells, is that the cells simply aren't thick enough to use these rather high energy photons. This means that the energy of the photons hitting the cells is just so high that they pass right through the cell without or just barely interacting with the atoms in the solar cell hence no photocurrent is produced and the quantum efficiency falls for these wavelengths. \newline
Also there is a connection betweeen the way the quantum efficiency falls for wavelengths approaching the energy gap of the absorber material and the kind of energy gap. First of all the quantum efficency falls in this range of wavelengths because now their energy is exactly the one needed to produce electron-hole pairs in the absorber material, for instance the silicon the solar cell is made of. Because these electron-hole pairs are produced outside of the depletion region there is no electric field to seperate them and therefore the recombine, not contributing to the photocurrent and therefore lowering the quantum efficiency. The way the kind of energy gap and the way the quantum efficiency drops can be explained using figure \ref{fig:absorption}. What you can see here is the absorption coefficient for various materials. While the absorption coefficient for materials with a direct energy gap, like GaAs, abruptly rises as soon as the photon energy reaches the energy of the gap, it rises more slowly for materials with an indirect energy gap just like crystalline silicon. This directly corresponds to the drop in the quantum efficiency, so for materials with an direct gap this drop is much more abruptly than for materials with an indirect gap. 
\begin{figure}[h]
	\centering
	\includegraphics[width=0.7\linewidth]{bilder/Absorption}
	\caption{Wavelenght depedence of the absorption coefficient for various materials}
	\label{fig:absorption}
\end{figure}
